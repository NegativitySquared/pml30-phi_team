
\subsubsection{13.10.14}

\begin{enumerate}
	\item Время начала и окончания собрания:\newline
	21:00 - 21:30
	\item Цели собрания:\newline
	\begin{enumerate}
	  \item Установить направляющие на робота.\newline
	  
	  \item Разработать концепцию механизма для закидывания мячей в подвижные корзины.\newline
	  
    \end{enumerate}
	\item Проделанная работа:\newline
	\begin{enumerate}
	  \item Поскольку в нашу стратегию входило возить подвижную корзину за собой сзади, было решено установить на последней направляющей ось, вокруг которой мог бы вращаться ковш с мячами. Сверху ковша будет помещена трубка диаметром чуть больше диаметра большего шарика, которая будет поворачиваться вокруг оси вместе с ним и в тот момент, когда ковш будет находиться выше трубки, мячи по ней будут скатываться назад, за робота, туда, где расположена корзина.\newline
      
      \item  Было подсчитано, что оптимальным местом расположения оси, вокруг которой должен будет вращаться ковш, является помещение ее в 20 сантиметрах от нижнего края последнего сегмента направляющих. Дополнительный прирост высоты подъема ковша позволил отказаться от одной пары 30-сантиметровых мебельных реек. Таким образом, у нас осталось три пары мебельных реек рабочей высотой в 105 сантиметров и механизм для закидывания мячей в корзины, находящийся на стадии разработки.\newline
      
      \item  Измененные направляющие были установлены на робота.\newline
      
      \item  После установки направляющих было решено протестировать работу подъемника. Ремень хорошо справлялся с задачей, однако оси прогибались, испытывая сильные нагрузки, из чего был сделан вывод, что нам необходимы более прочные оси. Кроме того, рейки раздвигались неодинаково, поэтому было решено попарно жестко скрепить  их друг с другом. Критических проблем в системе подъемника обнаружено не было.\newline
      
    \end{enumerate}
    
	\item Итоги собрания: \newline
	\begin{enumerate}
	  \item  Направляющие установлены на робота. Их конструкция упрощена.\newline
	  
      \item  Разработана концепция механизма закидывания мячей в корзины.\newline
    \end{enumerate}
    
	\item Задачи для последующих собраний:\newline
	\begin{enumerate}
	  \item Собрать устройство для вращения ковша.\newline
	  
	  \item Найти и купить более прочные оси для подъемника.\newline
	  
	  \item Купить еще один алюминиевый профиль для жесткого закрепления мебельных направляющих между собой.\newline

    \end{enumerate}     
\end{enumerate}

\fillpage
