\subsubsection{12.02.15 (Соревнования)}
\begin{center}
	2-ой день соревнований "Робофест-7 в Москве"
\end{center}
Сегодня проходили квалификационные матчи. \newline

Перед квалификационными матчами мы провели технический осмотр робота и подкрутили все разболтавшиеся винты и фиксаторы втулок на осях. Никаких изменений в конструкции за сегодняшний день реализовано не было.\newline

Изначально организаторами было запланировано провести 40 матчей так, чтобы каждая команда сыграла с каждой по 5 раз, но из-за неполадок с системой управления игровым полем, было сыграно только 32 игры, то есть каждая команда отыграла по 4.\newline
 
Результаты матчей: 3 победы из 4.\newline

Победные игры мы завершили с хорошим счетом. В каждой игре мы получали в среднем 50 очков за автономный период (из задуманных действий - закидывания мячей в две корзины и завоза корзин в зону парковки - за игру выполнялось 2 или 3, но каждый раз различные, что говорит о неточности движения с помощью энкодеров), захватывали подвижную корзину 90 см  и забрасывали в нее 2 раза по 4-5 больших мячей (это давало около 60 см высоты, то есть 180 очков), затем оставляли подвижную корзину в зоне парковки (10 очков), поскольку завозить полную корзину на пандус было опасно: существовал риск уронить ее и потерять все очки. После оставления корзины в зоне мы собирали мячи для центральной корзины и в финале забрасывали в нее от 1 до 3 больших мячей. То есть, мы следовали продуманной ранее стратегии. \newline

Во второй игре наш альянс набрал довольно большую сумму (по сравнению с матчами других команд) в 495 очков. Этого удалось добиться благодаря тому, что мы добились качественного командного взаимодействия с нашим союзником по альянсу - командой из Рязани под номером 22. Наши союзники не могли точно забрасывать мячи в подвижные корзины, поэтому мы сначала откидывали направляющую для мячей ровно над корзиной, а затем наши союзники забрасывали мячи в нее. Благодаря такой стратегии, совместными усилиями мы сумели заполнить корзину 90 см доверху. После этого мы забросили в центральную корзину 3 больших мяча и один маленький и заехали в зону парковки, а наш союзник затолкал одну из оставшихся подвижных корзин на пандус и заехал на него сам.\newline

В третьей игре мы проиграли, так как, во-первых, забыли положить в робота автономные мячи и не получили за автономный период очков, и кроме того робот по непонятным причинам не повернул в зону парковки после захвата подвижных корзин, а поехал прямо  и врезался в робота соперника, которому не хватало мощности сбить палку-подпорку и палка упала, что дало противнику дополнительное преимущество в 30 очков, а во-вторых, в управляемом сделали много ошибок, в связи с чем заполнили только 30 см в 90-сантиметровой корзине и не успели забросить мячи в центральную корзину.\newline

Начиная с этих соревнований, порядок проведения защиты технической книги изменился. Теперь судьи сами подходили к участникам во время проведения квалификационных матчей и общались с ними в форме диалога. Данный формат позволял судьям получить более полное представление о том, насколько хорошо участники понимают, что они делают и наколько продуманы были их технические решения. Для нас большим плюсом данного формата являлось то, что нам было не нужно учить речь для презентации нашего робота, как мы это делали раньше. Кроме того, теперь мы могли наглядно продемонстрировать возможности нашего робота судьям на тренировочном поле. В целом, мы хорошо справились с защитой книги. Результаты защиты инженерной книги пока неизвестны.\newline

Основные проблемы, выявленные в ходе матчей:
\begin{enumerate}
	\item Из-за того, что наш захват мячей был узким и, следовательно, захватывал мяч только в узком секторе, мы испытывали трудности с прицеливанием к мячам для их взятия. 
	
	\item Во время матчей некоторые команды в автономном периоде сдвигали подвижные корзины противоположного альянса с начальных позиций или блокировали к ним доступ. Это делало невозможным реализацию автономного периода, связанного с закидыванием мячей в подвижные корзины и перемещением корзин в зону парковки. Несмотря на то, что нам за все время квалификационных матчей команда с такой тактикой не попалась, мы поняли уязвимость своего автонома.
	
	\item После квалификационных матчей мы продолжили тренироваться и заметили, что ковш больше не может опрокидывать корзину с пятью мячами. Поскольку ранее он мог это делать, мы пришли к выводу, что он частично испортился от больших нагрузок. Нового мощного сервопривода в распоряжении у нас нет, поэтому в следующих играх (если мы пройдем в финал) нам придется закидывать в корзины по 4 больших мяча.
	
\end{enumerate} 

\fillpage