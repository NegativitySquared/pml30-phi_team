\subsubsection{23.11.14 (Соревнования)}
\begin{center}
	2-ой день соревнований "Робофест-Рязань"
\end{center}
Сегодня проходили квалификационные и финальные матчи, а также защита инженерных книг.
\newline 

Основные проблемы, выявленные в ходе матчей:
\begin{enumerate}
	\item Часто отходят провода соединяющие сервопривод, опрокидывающий ковш, с сервоконтроллером в местах их соединения друг с другом.
	
	\item Во время одного из матчей провода соединяющюе драйвера моторов с аккумулятором замыкались друг на друге.
	
	\item В ковше периодически возникали трещины из-за ударов о припятствия (например, центральную корзину).
	
\end{enumerate}

Результаты соревнований:
\begin{enumerate}
	\item По результатам квалификационных матчей в топ-2 мы не попали.
	
	\item В финальные игры мы вышли, поскольку были выбраны командой "ФМЛ№30 ${\psi}$", набравшей наибольшее количество очков по результатам квыалификационных матчей, и в результате наш альянс занял первое место.
	
	\item Мы заняли 1 место в номинации "Защита инженерной книги".
\end{enumerate}

Подведение итогов:
\begin{enumerate}
  \item Успешность выступления на соревнованиях:
  \begin{enumerate}
	\item По результатам игры наш альянс занял первое место, что является в большей части заслугой нашего союзника по альянсу - команды "ФМЛ№30 ${\psi}$".
	
	\item Мы заняли первое место в категории "Защита инженерной книги".
	
  \end{enumerate}
  
  \item Наши ошибки и недостатки конструкции:
  \begin{enumerate}
  	\item Проводка была ненадежна, из-за чего часто отказывали сервоприводы, отвечающие за опрокидывание ковша, а также приводы движения.
  	
  \end{enumerate}
  
  \item Задачи для последующих собраний:
  \begin{enumerate}
  	\item Целиком заменить проводку на более надежную.
  	
  	\item Реализовать более удобную программу управления движением робота.
  	
  	\item Изготовить ковш из качественного листового пластика.
  	
  	\item Увеличить высоту боковых откосов в системе захвата мячей.
  	
  	\item Демонтировать болт, затормаживающий лопасти.
  	
  	\item Реализовать МЗК, способный удерживать корзину после отключения питания.
  	  	
  	\item Установаить на МЗК откосы для центровки подвижной корзины, изготовленные из пластмассовой бутылки.
  	
  \end{enumerate}
  
\end{enumerate}
\fillpage