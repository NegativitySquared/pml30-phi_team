
\subsubsection{24.10.14}

\begin{enumerate}
	\item Время начала и окончания собрания:\newline
	16:00 - 20:00
	\item Цели собрания:\newline
	\begin{enumerate}
	  \item Устарнить баг сервопривода.\newline
	  
	  \item Изготовить из приобретенного нами листа алюминия и установить откосы для мячей в виде балок, расположенных в форме воронки в передней части робота.\newline
	  
    \end{enumerate}
    
	\item Проделанная работа:\newline
	\begin{enumerate}
	  \item Баг сервопривода устарнен. Причина его возникновения- вращение сервопривода после остановки с небольшой скоростью, которой не хватало на преодоление силы упругости стяжек. Связано это с нерпавильным значением мощности сервопривода в коде (значение, в котором сервопривод не вращается - 127 вместо стоявшем у нас 135).\newline
      
      \item Лист алюминия распилен на полосы нужной длины и ширины.\newline
      
      \item Откосы установлены на робота и протестированы. Результат положительный.\newline
      
      \item При тестировании откосов было замечено, что при столкновении с жестким препятствием они изгибаются. Для предотвращения этого были установлены упоры выпиленные из алюминиевой полосы.\newline
      
      \item Подготовлены отверстия для установки оставшейся пары поперечных перекладин на подъемнике.\newline
      
    \end{enumerate}
    
	\item Итоги собрания: \newline
	\begin{enumerate}
	  \item Баг сервопривода устранен.\newline
	  
      \item Откосы для мячей установлены на робота.\newline
      
    \end{enumerate}
    
	\item Задачи для последующих собраний:\newline
	\begin{enumerate}
	  \item Спроектировать и создать механизм захвата передвижных корзин.\newline
	  
    \end{enumerate}     
\end{enumerate}

\fillpage
