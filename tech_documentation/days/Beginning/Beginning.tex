\section{Инженерный раздел}
\subsection{Концепция робота}
\subsubsection{Конструкция}
\begin{itemize}
	\item Робот должен быть мобильным, двигаться быстро и, по возможности, в любом направлении (имеется в виду способность двигаться боком).
	\item Робот должен быть снабжен датчиками угла оборота моторов (энкодерами) для лучшей управляемости в автономном периоде.
	\item Робот должен быть компактным и не занимать лишнего места, чтобы не мешать союзнику по альянсу, а также для удобства транспортировки.
	\item Робот должен быть способен контролировать пять (5) мячей одновременно.
	\item Робот должен иметь приспособление для перемещения подвижных корзин.
	\item По возможности, робот должен быть легким, чтобы его было легче переносить.
	\item Конструкция робота должна обеспечивать быстрый доступ ко всем его ключевым узлам.
\end{itemize}
\subsubsection{Автономный период}
\begin{itemize}
	\item Робот должен иметь несколько версий автономного периода, в зависимости от того, где он стартует и каковы возможности союзника по альянсу.
	\item Программа автономного периода должна быть, по возможности, простой.
	\item По возможности, должна быть реализована программа, дающая роботу возможность ориентироваться по ИК-датчику.
\end{itemize}
\subsubsection{Управляемый период}
\begin{itemize}
	\item Управление роботом должно быть простым, удобным и интуитивно понятным.
	\item Один оператор полностью отвечает за перемещение робота, а второй - за все остальные функции.
	\item Некоторые действия в управляемом режиме могут быть осуществлены автономно, для того, чтобы снять лишнюю задачу с оператора.
	\item Желательно, чтобы у оператора была возможность управления скоростью робота, поскольку совершать точные манипуляции с игровыми элементами на максимальной скорости не рационально.
\end{itemize}
\fillpage

\subsection{Стратегия} 
\subsubsection{Автономный период}
\begin{enumerate}
    \item Положить два автономных мяча в две разные корзины (подвижные либо центральную).
	\item Взять максимальное количество передвижных корзин и отвезти их в зону парковки.
	\item По пути в зону парковки  задействовать механизм высвобождения мячей.
	         
\end{enumerate}
\subsubsection{Управляемый период-  основная часть}
\begin{enumerate}
	\item Обеспечить свободный доступ союзника по команде к подвижным корзинам. Но, при этом, возить за собой одну корзину, чтобы не тратить много времени на перемещение мячей в неё.
	\item Наполнить мячами сначала 90-сантиметровую корзину, затем - 60-сантиметровую и 30-сантиметровую.
	\item Избегать столкновений, как с союзником, так и с противниками, поскольку из-за этого теряется время.
\end{enumerate}
\subsubsection{Управляемый период - финал}
\begin{enumerate}
	\item Заполнить центральную корзину большими мячами.
	\item Отвезти максимально возможное количество передвижных корзин на пандус.
	\item Заехать роботом на пандус. 
\end{enumerate}
\fillpage

\subsection{Планируемые этапы создания робота}
\begin{enumerate}
	\item Создание колесной (или гусеничной) базы робота.
	\item Написание программы для управления колесной базой с помощью одного (1) джойстика.
	\item Создание системы контроля мячей.
	\item Написание программы для управления системой контроля мячей параллельно с движением робота с помощью двух (2) джойстиков.
	\item Написание программы для автономного периода.
	\item Создание дополнительных декоративных элементов.
	\item Установка на робота дополнительной защиты корпуса для предотвращения повреждений при случайных столкновениях.
	\item Тренировки (в одиночку или с другими роботами ).
	\item Внесение доработок по итогам первых соревнований.
\end{enumerate}
\fillpage
