\subsubsection{18.11.14}

\begin{enumerate}
	\item Время начала и окончания собрания:
	16:00 - 1:00
	\item Цели собрания:
	\begin{enumerate}
	  \item Тренироваться в управлении роботом
	  
    \end{enumerate}
	\item Проделанная работа:
	\begin{enumerate}
	  \item Для более стабильного выкатывания шариков из ковша было решено заменить обрезки от пластиковых бутылок на металлическую сетку. Это позволило увеличить диаметр трубы ковша
      
      \item В процессе тренировок было обнаружено, что сервопривод, опрокидывающий ковш не способен повернуть его, когда он заполнен. Для исправления этого на заднюю часть ковша был установлен противовес в виде груза 
      
      \item Также было обнаружено, что 2 мотора не справляются с раздвиганием подъемника, из-за чего предохранители начинают дымиться. Для исправления этого была установлена понижающая передача
      
      \item После установки передачи моторы все равно не спралялись с раздвиганием подъемника. Связано это с тем, что в какой-то момент ремень начинает зажимать между верхней поперечной осью самой нижней рейки и нижней осью второй рейки. Во избежание этого нужно поставить ограничители, которые будут не давать нижней оси второй рейки подниматься слишком высоко
          
    \end{enumerate}
    
	\item Итоги собрания: 
	\begin{enumerate}
	  \item Обрезки пластиковой бутылки на ковше заменены на металлическую сетку 
	  
      \item На моторы, раздвигающие подъемник, установлена понижающая передача
      
      \item На ковш установлен противовес в виде груза
    \end{enumerate}
    
	\item Задачи для последующих собраний:
	\begin{enumerate}
	  \item Установить ограничители на оси
	  
	  \item Тренироваться в управлении роботом

    \end{enumerate}     
\end{enumerate}
\fillpage